\subsection{Ideas and Technologies}
\subsubsection{Technologies to be Used}
After due consideration and taking into account the received requirements, the following technologies will be used:
\begin{description}
	\item[ASP.NET MVC 5]\hfill\\
	%Description
	This will be used to develop the back-end of the system, using Microsoft's Visual Studio IDE.
	\item[ASP.NET Web API]\hfill\\
	This will form the back-bone of the API with which the rest system will interact (possibly hosted in the cloud).
	\item[C\#]\hfill\\
	Will be used in conjunction with .NET development.
	\item[MySQL]\hfill\\	
	This is the relation database which will be used.
	\item[Xamarin or Android Studio]\hfill\\
	The decision has yet to be made on whether the C\# Xamarin approach will be used or the traditional Java in Android Studio approach, to develop an Android application. One of these two will however reign supreme.
	
\end{description}
	
\subsubsection{Implementation Ideas}	
In order to develop the back end of the system [API]:\\
- One could do hosting in the cloud with the Micrsoft Azure hosting platform
- As students we would have access to many Microsoft products by registering at \href{http://www.dreamspark.com}{www.dreamspark.com}\\

The Repository Pattern as well as IoC (Inversion of Control) can be implemented. Another possible pattern that could be investigated is the Unit of Work pattern.\\

Open authentication using Gmail or Facebook, can be used to login and compare driving scores or results, either on the website or a social network.\\

There are various Android Wear APIs which form part of the Android Support Library and Google Play services. When using these libraries, hand-held devices running Android 4.3 or later can communicate with wearables. These APIs can handle synced notifications, voice actions and sending data between the hand-held devices and wearables devices.
	
\subsubsection{Project Extension Ideas}
The following was considered as possible future extensions on the current project specifications:

\begin{itemize}
	\item Driving Tips\\
	Tips could be given by the app on to improve a certain individuals driving e.g. brake slower, as to increase his score.
	\item Android watch notification sister-application\\
The idea behind this is that while the main app is tracking the user's travelling speed, if it is the case that a user is travelling at a rate which is greater than the speed limit in his vicinity the user will be made aware of this fact. This alert can take place in the form of vibrations or sound emitted by the watch as to not interfere with the user's driving.
	\item Voice Recognition\\
	CMUSphinx can be used as the speech recognition software. It is open source and it also provides good Java integration and demo applications. The alerts from the sister-application could be dismissed by speaking a certain phrase as an example.\\
	\item Game mode for non-business purposes\\
	By providing a "game-mode", it could do a lot for the popularity of the app if users could challenge their friends to see who is the better driver.
	\item Social media\\
The app could be linked to social media – where users compare driving ratings with other members in their friend list.
\end{itemize}